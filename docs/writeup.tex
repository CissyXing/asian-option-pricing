\documentclass{article}
\usepackage{mathtools}
\usepackage{indentfirst}
\usepackage{url}

\begin{document}

\tableofcontents

\newpage

\section{Introduction}

The Asian Option is an option whose payoff is dependent on the average price of the underlying asset over a period of time. These come in fixed strike and floating strike variants. Many different methods of pricing exist, among which include closed-form approximations, lattice methods, as well as Monte Carlo simulations. Each have distinct advantages over the other.

In this paper, we price an Asian option using the Implicit Finite Difference Scheme method, applying transformations used by 5 different Partial Differential Equations to construct the Finite Difference Grid.

\section{Definitions}
The following variables will be used throughout the paper. Where necessary, mathematical definitions will follow later.
\begin{table}[h]
  \begin{tabular}{|c|c|}
    \hline
    \textbf{Description} & \textbf{Variable Name} \\ \hline
    Stock Price & \(S\) \\
    Strike Price & \(K\)\\
    Average Price & \(A\) \\
    Time ago which averaging started & \(t_0\) \\
    Remaining lifespan of the Option & \(T\) \\
    Risk Free Rate & \(r\) \\
    Dividend Rate & \(\delta\) \\
    \hline
  \end{tabular}
  \caption{Variables used in various equations}
  \label{table:name}
\end{table}

\section{Finite Difference Schemes}

The Finite Difference Scheme method allows for flexibility between accuracy and time. Monte Carlo methods are heavily dependent on the number of paths to increase accuracy \cite{montecarlo}. For closed form approximations, no tradeoff can be made.

The Implicit Finite Difference Scheme is a modification of the Binomial Tree Method, whereby predefining a grid and setting particular boundary and terminal values allows us to iteratively recover the fair price of an option today.

INSERT FDS DIAGRAMS HERE

\section{Derivation}

The payoff of the options for the fixed and floating strike calls are respectively \(max\{A_T - K, 0\}\) and \(max\{S_T - A_T, 0\}\), given the average price of a stock, which can be calculated by
\begin{equation}
  A_T = \frac{t_0A + \int_0^T S_t \mathrm{d}t}{t_0 + T}.
\end{equation}.

Supposing that the stock price \(S\) follows Geometric Brownian Motion, we have

\begin{equation}
  \mathrm{d}S_t = \mu S_t \mathrm{d}t + \sigma S_t \mathrm{d}W_t.
\end{equation}

In the risk-neutral world, this comes to
\begin{equation}
  \mathrm{d}\tilde{S}_t = \mu \tilde{S}_t \mathrm{d}t + \sigma \tilde{S}_t \mathrm{d}W_t,
\end{equation}
where \(\tilde{S}_t = e^{-\delta(T-t)}S_t\).

For convenience purposes, we also define the function \(q(t)\) such that
\begin{equation}
  q(t) =
  \begin{dcases*}
    1 - e^{-(r-\delta)(T-t)} & \(r \neq \delta\) \\
    \frac{T-t}{t_0 + T} & \(r\) = \(\delta\)
  \end{dcases*},
\end{equation}

We obtain the value process of a self-financing portfolio with payoff \(A_T\), also known as the average asset in our situation here, using the following identity:

\begin{equation}
  E_P(\int_0^TS_u\textrm{d}u | \mathcal{F}_t) = \int_0^t S_u \textrm{d}u + \int_t^T S_te^{(r-\delta)(u-t)}\textrm{d}u,
\end{equation}

\begin{equation}
  A_t = e^{-r(T-t)}E_P(A_T|\mathcal{F}_T) = q(t)\tilde{S_t} + e^{-r(T-t)} \frac{t_0 A + \int_0^tS_u\mathrm{d}u}{t_0+T}.
\end{equation}

Using Ito's Lemma on the risk-neutral asset yields

\begin{equation}
 \textrm{d}(\frac{e^{rt}}{\tilde{S}_t}) = -\sigma \frac{e^{rt}}{\tilde{S}_t}\textrm{d}W_t^S,
\end{equation}
where \(W_t^S = W_t - \sigma t\)

We also define
\begin{equation}
  \xi_t = a(t) + b(t) \frac{A_t - Ke^{-\delta(T-t)}}{\tilde{S_t}}.
\end{equation}

and are given the following governing partial differential equation

\begin{equation}
  \phi_t + [\dot{a}(t) + \dot{b}(t)(\xi - a(t))/b(t)]\phi_\xi + \frac{1}{2}\sigma^2[a(t)+b(t)q(t) - \xi]^2\phi_{\xi\xi} = 0,
\end{equation}

where \(\phi(T, \xi) = f(\xi)\), and \(f(\xi)\) is defined as per the below table.

\begin{table}[h]
  \begin{tabular}{|c|c|c|}
    \hline
    & Fixed Strike & Floating Strike \\
    \hline
    Calls & \(max\{(\xi - a(T))/b(T), 0\}\) & \(max\{(a(T) - \xi)/b(T) + 1, 0\}\) \\
    Puts & \(max\{(a(T) - \xi)/b(T ), 0\}\) & \(max\{(\xi - a(T))/b(T) - 1, 0\}\)\\
    \hline
  \end{tabular}
  \caption{Different possible values of \(f(\xi)\)}
\end{table}

\section{Partial Differential Equations}

\subsection{Boundary Conditions}
\begin{table}[h]
  \begin{tabular}{|c|c|c|}
    \hline
    & \(\xi \rightarrow \infty\) & \(\xi \rightarrow -\infty\) \\
    \hline
    \(b(t) > 0\) &  & \(\phi(t, \xi) \rightarrow 0\) \\
    \(b(t) < 0\) & \(\phi(t, \xi) \rightarrow 0\) & \\
    \hline
  \end{tabular}\\
  \begin{tabular}{|c|c|c|}
    \hline
    & \(\xi \ge a(t) + q(t)b(t)\) & \(\xi \le a(t) + q(t)b(t)\) \\
    \hline
    \(b(t) > 0\) & \(\phi(t, \xi) = \frac{\xi-a(t)}{b(t)} \) & \\
    \(b(t) < 0\) &  & \( \phi(t, \xi) = \frac{\xi-a(t)}{b(t)} \) \\
    \hline
  \end{tabular}
  \caption{Different Boundary Conditions}
\end{table}

\subsection{Rogers Shi Fixed Strike Calls}
We take
\begin{equation}
  \xi_t = q(t)e^{(r-\delta)(T-t)} - e^{(r-\delta)(T-t)}\frac{A_t - Ke^{-\delta(T-t)}}{\tilde{S_t}}.
\end{equation}

Then, we have \( a(t) = -q(t)b(t) \) and \(b(t) = -e^{-(r-\delta)(T-t)}\). The partial differential equation then becomes

\begin{equation}
  \phi_t - ( (r-\delta)\xi + \frac{1}{t_0 + T} ) \phi_\xi + \frac{1}{2}\sigma^2\xi^2\phi_{\xi\xi} = 0.
\end{equation}

Rogers and Shi show that the value of the option at time \(t\) is given by \(S_t\phi(t, \xi_t)\). We then solve the PDE with the following conditions:

\begin{equation}
  \phi(T, \xi) = 0, 0 \le \xi \le \xi_{max}
\end{equation}
\begin{equation}
  \phi(t, 0) = q(t) \\
\end{equation}
\begin{equation}
  \phi(t, \xi_{max}) = 0, 0 \le t \le T.
\end{equation}

This problem is transformed with \(\tau = T - t \), such that \(\psi(\tau, \xi) = \phi(T-\tau, \xi)\). This changes our partial differential equation to
\begin{equation}
  \psi_\tau = \frac{1}{2}\sigma^2\xi^2\psi_{\xi\xi} - (r\xi + \frac{1}{T})\psi_\xi,
\end{equation}
subject to the conditions
\begin{equation}
  \psi(0, \xi) = 0, 0 \le \xi \le \xi_{max},
\end{equation}
\begin{equation}
  \psi(\tau, 0) = \frac{1-e^{-r\tau}}{rT},
\end{equation}
\begin{equation}
  \psi(\tau, \xi_{max}) = 0, 0 \le \tau \le T.
\end{equation}

We let \(\Delta\tau = \frac{T}{n}\) and \(\Delta\xi = \frac{\xi_{max}}{m}\), and define \(0 \le i \le n\) and \(0 \le j \le m\) as the indexes required for the finite difference grid

Replacing the PDE with the Crank-Nicholson discretizations yields the following equation:
\begin{equation}
  \begin{split}
    \frac{u_{i+1, j} - u_{i, j}}{\Delta\tau} = & \frac{1}{2}\sigma^2j^2(\Delta\xi)^2 * \frac{u_{i, j+1} - 2u_{i, j} + u_{i, j-1} -2u_{i+1, j} + u_{i+1, j-1}}{2(\Delta\xi)^2} \\ & - (rj\Delta\xi + \frac{1}{T}) * \frac{u_{i, j+1} - u_{i,j-1} +u_{i+1, j+1} - u_{i+1, j-1}}{4\Delta\xi}.
  \end{split}
\end{equation}

We extract out the common variables such that
\begin{equation}
  \alpha(j) = \frac{1}{4}\sigma^2j^2\Delta\tau,
\end{equation}
\begin{equation}
  \beta(j) = \frac{1}{4\Delta\xi}\Delta\tau(rj\Delta\xi - \frac{1}{T}).
\end{equation}

The discretization is rearranged to yield
\begin{equation}
  \begin{split}
    u_{i+1, j} = & (1-2\alpha(j))u_{i, j} \\
    & + (\alpha(j)-\beta(j))u_{i, j+1}\\
    & + (\alpha(j)+\beta(j))u_{i, j-1}\\
    & + (-2\alpha(j))u_{i+1, j}\\
    & + (\alpha(j)-\beta(j))u_{i+1, j+1}\\
    & + (\alpha(j)+\beta(j))u_{i+1, j-1}.
  \end{split}
\end{equation}

We convert to the matrix form required for the Implicit Difference Scheme used. Truncating the rows where \(j=0\) and \(j=m\), we have

\begin{equation}
  \textbf{r} = \textbf{Al} + \textbf{Br} + \textbf{k},
\end{equation}

where

\begin{equation}
  \textbf{l} = u_{i},
\end{equation}
\begin{equation}
  \textbf{r} = u_i+1,
\end{equation}
\begin{equation}
  \textbf{A} = \begin{bmatrix}
    1-2\alpha(1) & \alpha(1) - \beta(1) & 0 & 0 & \hdots & 0 \\
    \alpha(2) + \beta(2) & 1-2\alpha(2) & \alpha(2) - \beta(2) & 0 & \hdots & 0 \\
    0 & \alpha(3) + \beta(3) & 1-2\alpha(3) & \alpha(3) - \beta(3) & \hdots & 0 \\
    \vdots & \vdots & \vdots & \vdots & \ddots & \vdots \\
    0 & 0 & 0 & 0 & \hdots & 1-2\alpha(m-1)
  \end{bmatrix},
\end{equation}
\begin{equation}
  \textbf{B} = \begin{bmatrix}
    -2\alpha(1) & \alpha(1) - \beta(1) & 0 & 0 & \hdots & 0 \\
    \alpha(2) + \beta(2) & -2\alpha(2) & \alpha(2) - \beta(2) & 0 & \hdots & 0 \\
    0 & \alpha(3) + \beta(3) & -2\alpha(3) & \alpha(3) - \beta(3) & \hdots & 0 \\
    \vdots & \vdots & \vdots & \vdots & \ddots & \vdots \\
    0 & 0 & 0 & 0 & \hdots & -2\alpha(m-1)
  \end{bmatrix},
\end{equation}
\begin{equation}
  \textbf{k} = \begin{bmatrix}
    (\alpha(1) + \beta(1))u_{i, 0} + (\alpha(1) + \beta(1))u_{i+1, 0} \\
    0 \\
    \vdots \\
    0 \\
    (\alpha(m-1) - \beta(m-1))u_{i, m} + (\alpha(m-1) + \beta(m-1))u_{i+1, m}
  \end{bmatrix}
\end{equation}

This rearranges to

\begin{equation}
  \textbf{r} = (\textbf{I} - \textbf{B})^{-1}(\textbf{Al}+\textbf{k}).
\end{equation}

We can then proceed with solving the Finite Difference Grid based on the above recurrence relation. Adjusting \(\xi_{max}\) such that we have

\begin{equation}
  \xi_{max} = \frac{3K}{S_0}.
\end{equation}

We select \(j'\) such that

\begin{equation}
  j' = round(\frac{K}{S_0\Delta\xi}).
\end{equation}

Then, the fair price of the option \(P\) is given by

\begin{equation}
  P = S_0 u_{n, j'}
\end{equation}

\section{Implementation}

\subsection{Inheritance and Overriding}
Since the methods governing the solutions of the various Partial Differential Equations are largely similar, we adopt the object-oriented approach. Inheritance and overriding are fundamental concepts of object-oriented programming, and reduce the complexity of the required implementation.

Inheritance enables new objects to take on the properties of existing objects \cite{oop_inheritance}. In this case, we define an \texttt{Option} superclass which implements all commonalities between both fixed and floating strike calls, including keeping track of properties, setting up the grid as well as some of the common methods. \texttt{FixedCall} and \texttt{FloatingCall} superclasses are also defined to handle the mild differences between the two cases.

Overriding is a feature that enables a child class to provide different implementation for a method that is already defined and/or implemented in its parent class or one of its parent classes \cite{oop_override}. This allows us to implement abstract methods defaulting to a non-damaging value and override them in a subclass only if necessary.

The overall advantage of this design allows for the reduction of 10 different algorithms into a more centralized structure where each algorithm implements only the minimum requirement. The language of choice was Python - a free programming language. We also heavily use the \texttt{NumPy} package, which is the fundamental package for scientific computing with Python.

\subsection{\texttt{Option} superclass}

\subsection{\texttt{FixedCall} and \texttt{FloatingCall}}

\subsection{A minimal example implementation}

\bibliographystyle{unsrt}
\addcontentsline{toc}{section}{Bibliography}
\bibliography{ref}

\end{document}
